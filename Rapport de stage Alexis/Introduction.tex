\hypertarget{Introduction}{%
\chapter{Introduction}\label{Introduction}}

\section{Présentation de la structure
d'accueil}

Durant la période de mon stage , j'ai été accueilli au
\textbf{Laboratoire de Mathématiques Informatique et Application
(LAMIA)} de l'Université des Antilles (UA).

Pour présenter cette structure, il me faut tout d'abord présenter
l'université à laquelle il est rattaché.

\hypertarget{luniversite-des-antilles}{%
\subsection{L'université des Antilles}\label{luniversite-des-antilles}}

Bien que ce soit l'université dans laquelle j'ai fait toutes mes études,
voici quelques chiffres que je ne connaissais pas et qui donnent la
mesure de sa taille :

L'Université des Antilles s'organise autour deux pôles universitaires
régionaux autonomes : le « Pôle Guadeloupe » et le « Pôle Martinique ».

Sur ces pôles, l'Université assure des missions d'\emph{enseignement} et
de \emph{recherche}, assistées par des \emph{administratifs et des
techniciens}.

\hypertarget{administration-et-personnel-technique}{%
\subsubsection{Administration et personnel
technique}
\label{administration-et-personnel-technique}}

l'UA emploie 414 Administratifs et Techniciens (environ 200 personnes
pour l'administation centrale et 100 répartis sur chaque pôle)

\hypertarget{enseignements}{%
\subsubsection{Enseignements}\label{enseignements}}

L'UA délivre des diplomes de la licence au doctorat dans de nombreux
domaines. Au total, cela représente :

\begin{itemize}
\tightlist
\item
  484 enseignants-chercheurs (environ 240 pour chaque pôle)
\item
  12 000 étudiants (environ 7000 pour la Guadeloupe , 5000 pour la
  Martinique)
\end{itemize}

Pour l'informatique, cela représente : - autour de 20
enseignants-chercheurs - autour de 120 étudiants

\hypertarget{recherche}{%
\subsubsection{Recherche}\label{recherche}}

La recherche est structurée en laboratoires auxquels sont rattachés les
enseignants chercheurs qui peuvent former de futurs chercheurs : les
doctorants.

L'université compte ainsi au total :

\begin{itemize}
\tightlist
\item
  17 laboratoires
\item
  320 doctorants
\end{itemize}

Pour ma part, comme signalé précédemment, j'ai effectué mon stage dans
le laboratoire LAMIA que je vais maintenant présenter.

\hypertarget{le-lamia}{%
\subsection{Le LAMIA}\label{le-lamia}}

Le \textbf{Laboratoire de Mathématiques Informatique et Application
(LAMIA)}, comme son nom l'indique, se concentre sur les recherches en
informatiques et mathématiques.

Il compte une soixantaine de membres (Professeurs des Universités,
Maitres de Conférences, ATER, Doctorants) répartis sur deux pôles
(Guadeloupe et Martinique) au sein de trois équipes internes :

\begin{itemize}
\tightlist
\item
  Equipe
  \href{http://lamia.univ-ag.fr/index.php?page=equipe-mathematiques}{\textbf{Mathématiques}
  (analyse variationnelle, analyse numérique, EDP, analyse statistique,
  mathématiques discrètes)} ;
\item
  Equipe Informatique
  \href{http://lamia.univ-ag.fr/index.php?page=equipe-danais}{\textbf{DANAIS}
  : Data analytics and big data gathering with sensors} ;
\item
  Equipe Informatique
  \href{http://lamia.univ-ag.fr/index.php?page=equipe-aid}{\textbf{AID}
  : Apprentissages Interactions Donnees} ;
\end{itemize}

De plus, le LAMIA accueille en son sein un groupe de chercheurs associés
travaillant en Epidémiologie clinique et médecine.

L'équipe avec laquelle il m'a été donné de travailler principalement est
celle d'\textbf{Apprentissages Interactions Données} qui développe des
méthodes de traitements et d'analyse de données hétérogènes : images
(classique, multi-spectrale), séquences vidéos, séries temporelles et
spatio-temporelles, dont la responsable est \textbf{Mme. Hélène
Paugam-Moisy}.

Indépendamment de ces équipes, les travaux de recherche du laboratoire
se répartissent en \textbf{projets} qui peuvent réunir des membres de
plusieurs équipes en \textbf{groupes de travail}. Mon stage était en
fait plus attaché à un projet et un groupe de travail qu'à une équipe.

Ce projet est nommé de façon informelle projet \textbf{``Spikes''} et
concerne l'utilisation de \textbf{réseaux de neurones impulsionnels}
pour l'apprentissage automatique (ces notions seront définies plus
loin). Le groupe de travail associé réunit à l'heure actuelle :

\begin{itemize}
\tightlist
\item
  1 Professeur des Universités
\item
  2 MCF avec HDR
\item
  3 MCF
\item
  1 ingénieur d'études.
\end{itemize}

Le groupe accueille actuellement deux stagiaires (dont moi) , un Master
2 informatique et un Licence 3 informatique.

C'est avec ces personnes que j'ai travaillé tout au long du stage et mon
tuteur de stage était \textbf{Mr.~Vincent PAGÉ}.

Ci dessous, un schéma présentant la structure du laboratoire mon
rattachement à cette structure. (L'équipe de travail \textbf{Spikes}
étant informelle, elle ne figure pas sur ce schéma.)

\begin{figure}[h!]
\centering
\includegraphics[width=10cm]{./images/orga.jpg}
\caption{Figure 1 Schéma de l'organisation interne du LAMIA:
¹Apprentissages Interactions Données. ²Data analytics and big data
gathering with sensors. ³Mathématiques (analyse variationnelle, analyse
numérique, EDP, analyse statistique, mathématiques discrètes). ⁴Membres
permanents : Suzy Gaucher-Casalis~(MCF),~Enguerran
Grandchamp~(MCF--HDR),~Jean-Luc Henry~(MCF),~Jimmy Nagau~(MCF),~Vincent
Pagé(MCF),~Helene Paugam Moisy~(PR),~Sébastien Régis~(MCF),~Céline
Rémi~(MCF).}
\end{figure}

La prochaine section sera consacrée à la présentation de la thématique
de recherche du groupe ``Spikes'' et de mon stage.


\hypertarget{Objectif_Spike}{%
\section{Objectifs du groupe Spikes}\label{Objectif_Spike}}

Le groupe \textbf{Spikes} s'intéresse aux techniques
d'\textbf{Intelligence Artificielle}, plus spécifiquement à
l'\textbf{apprentissage automatique} dont l'objectif est de créer des
programmes capable d'apprendre à partir de bases d'exemples.

Actuellement, parmi les techniques permettant l'apprentissage
automatique , une se démarque et est très populaire : les
\textbf{réseaux de neurones artificiels} , notamment dans leur version
\emph{profonde} qui sont très utilisé par exemple par \textbf{Facebook™}
pour sa \textbf{reconnaissance faciale} ou encore par \textbf{Google™}
pour ses \textbf{robots} qui apprennent par \textbf{répétitions} à jouer
(échecs, Go).

\hypertarget{ruxe9seaux-de-neurones-classiques-et-impulsionnels}{%
\section{Réseaux de neurones classiques et
impulsionnels}\label{ruxe9seaux-de-neurones-classiques-et-impulsionnels}}

Ces réseaux de neurones artificiels sont nés dans les années 1960 et les
techniques d'apprentissage arrivent à maturité seulement maintenant,
avec des résultats très impressionnants.

Bien que basés de façon lointaine sur des neurones biologiques, ils sont
en fait très peu plausibles biologiquement (les neurones échangent des
chiffres entre eux, les informations se rétropropagent\ldots{})

A l'inverse, une autre génération de réseaux de neurones, dits
\textbf{réseaux de neurones impulsionnels} fait l'objet de nombreuses
recherches. Nous reviendrons sur le fonctionnement de ces réseaux plus
loin. Disons simplement que les neurones n'échangent plus des chiffres
mais des impulsions électriques datées, qui se propagent dans le réseau
et en déclenchent d'autres.

Ces réseaux de neurones seraient plus proches des neurones biologiques
et potentiellement capable de meilleurs résultats que les réseaux
classiques. Néanmoins, à l'heure actuelle, les performances en
apprentissage de ces réseaux de neurones sont beaucoup plus faibles que
celles des réseaux de neurones classiques.

L'activité de recherche du groupe \textbf{Spikes} consiste à
expérimenter de nouvelles techniques d'apprentissage pour ces
\textbf{réseaux de neurones impulsionnels} dans le but de contribuer à
les améliorer.

\hypertarget{problematique}{%
\section{Problématique: Diffusion de rumeur}\label{problematique}}

Mon travail consistait dans l'étude de la diffusion de rumeur dans un réseau social.
C'est un sujet très actuel et passionant. En effet,
une meilleure connaissance des mécanismes de transmission permettrait par exemple de lutter contre la diffusion de \textit{fake news} ou de promouvoir la propagation d'informations importantes.

Une étude datant de 2015~\cite{Lympero} montre que les réseaux de neurones impulsionnels pourraient permettre de modéliser la diffusion de rumeur.

Cette application des réseaux de neurones impulsionnels a semblé très intéressante au groupe \textbf{SpikeTrain}. Elle s'accordait en effet bien avec les tests actuels de ce groupe sur le simulateur de réseaux de neurones \textbf{Bindsnet}.

Dans ce cadre, mon stage consistait donc à essayer de reproduire les expériences menées dans cet article sur le simulateur \textit{Bindsnet}.

Tous ces concepts m'étant inconnus avant ce stage, il m'a fallu découvrir à la fois les réseaux de neurones impulsionnels, la diffusion de la rumeur ainsi que la façon dont ces réseaux peuvent modéliser la diffusion de rumeur. Ce sont tous ces points qui feront l'objet de la section suivante.
