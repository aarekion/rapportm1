\hypertarget{conclusion}{%
\chapter{Conclusion}\label{conclusion}}

Mettre un chapeau

\section{Travail réalisé}

Debrouille toi.

\section{Ce que SpikeTrain en tire}

L'analyse des fiches nous à permis de constater que :
\begin{itemize}
\item Certaines espèces semblent émettre à certaines gammes de fréquences
\item Contrairement à ce qui nous avait été annoncé dans la description du challenge de nombreux clics de la base de test n'étaient pas centrés.
\item Les signaux de la base d'apprentissage sont en très grande majorité bien centrés.
\item Un nombre important d'enregistrements de la base de test semblent avoir subbis d'importantes dégradations.
\item Sur certains enregistrements de la base de test le clic est impossible a identifier sans prétraitement.
\end{itemize}
Ce qui peux expliquer au moins en partie le différentiel de performances de nos IA sur la base labelisée et la base non labelisée.

En somme l'objectif est atteint.

\section{Ce qu'Alexis en tire}

Bien que les tâches auxquelles j'ai été affecté ne s'attardent que peu sur les
IA que nous avons utilisés, de par leur nature elle m'ont permis d'acquérir une
large gamme de compétences.

D'abord en programmation puisqu'avant de commencer mon stage je n'avais qu'un faible niveau en python et de maigres connaissances sur google colab, ainsi qu'aucune connaissance en LaTex, je ressors donc avec une certaine maitrise de ces trois langages.

Ensuite en terme de méthodologie, mon travail avec des chercheurs sur un sujet de recherche à distance m'a permis de m'initier et de m'exercer aux méthodes de recherche, de travail à distance.

J'ai pu aussi, apprendre à rédiger des documents de recherche grâce à la création des fiches d'analyse ainsi que la rédaction de ce rapport.

Les problémes techniques inhérents aux conditions exceptionnelles dans lesquelles mon stage s'est déroulé ont rendu son déroulement compliqué et laborieux.
Heureusement, la grande maîtrise et la remarquable fléxibilité de mes tuteurs en ont fait une expérience aussi intense qu'enrichissante.

\section{Perspectives}

Perspectives pour SpikeTrain :

- GUI pour l'outil notamment dans le cadre du projet H2020 déposé par le groupe spikeTrain.

- L'outil de création de fiche de visualisation me semble mature. En particulier,
il permet de rassembler en un seul fichier pdf, les visualisations de nombreux
exemples de la base. Il reste, pour mes tuteurs, à analyser finement ces visualisations pour améliorer les performances du classifieur.
Ils sont optimistes à ce sujet.

- A l'heure actuelle ces outils ne sont adaptés qu'à une base de données et à
un probléme particulier.
Par la suite, il devraient être transformés en outils génériques facilitant l'analyse des données des futurs problémes auxquels seront confrontés l'équipe SpikeTrain. Par exemple, ils comptent s'intéresser à de l'identification de
poissons dans les récifs coraliens en collaboration avec le laboratoire de
Biologie Marine de l'Université des Antilles.

Perspectives pour Alexis

- mon envie de faire de la recherche

- ton envie de faire de l'IA / signal / maths

- ton envie de faire des trucs
Actuellement l'ensemble des IA existantes donnant de bons résultats ont un point commun, elle demandent une quantité titanesque de données afin d'être efficaces.
Fort de ce constat les outils d'analyse et de tris des données permettant de naviguer facilement dans d'importantes quantités de données semble présenter un grand intérêt.
