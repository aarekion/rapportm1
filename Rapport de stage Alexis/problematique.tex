
\hypertarget{contexte-guxe9nuxe9ral}{%
\section{Contexte général}\label{contexte-guxe9nuxe9ral}}

Le groupe \textbf{Spikes} s'intéresse aux techniques
d'\textbf{Intelligence Artificielle}, plus spécifiquement à
l'\textbf{apprentissage automatique} dont l'objectif est de créer des
programmes capable d'apprendre à partir de bases d'exemples.

Actuellement, parmi les techniques permettant l'apprentissage
automatique , une se démarque et est très populaire : les
\textbf{réseaux de neurones artificiels} , notamment dans leur version
\emph{profonde} qui sont très utilisé par exemple par \textbf{Facebook™}
pour sa \textbf{reconnaissance faciale} ou encore par \textbf{Google™}
pour ses \textbf{robots} qui apprennent par \textbf{répétitions} à jouer
(échecs, Go).

\hypertarget{ruxe9seaux-de-neurones-classiques-et-impulsionnels}{%
\section{Réseaux de neurones classiques et
impulsionnels}\label{ruxe9seaux-de-neurones-classiques-et-impulsionnels}}

Ces réseaux de neurones artificiels sont nés dans les années 1960 et les
techniques d'apprentissage arrivent à maturité seulement maintenant,
avec des résultats très impressionnants.

Bien que basés de façon lointaine sur des neurones biologiques, ils sont
en fait très peu plausibles biologiquement (les neurones échangent des
chiffres entre eux, les informations se rétropropagent\ldots{})

A l'inverse, une autre génération de réseaux de neurones, dits
\textbf{réseaux de neurones impulsionnels} fait l'objet de nombreuses
recherches. Nous reviendrons sur le fonctionnement de ces réseaux plus
loin. Disons simplement que les neurones n'échangent plus des chiffres
mais des impulsions électriques datées, qui se propagent dans le réseau
et en déclenchent d'autres.

Ces réseaux de neurones seraient plus proches des neurones biologiques
et potentiellement capable de meilleurs résultats que les réseaux
classiques. Néanmoins, à l'heure actuelle, les performances en
apprentissage de ces réseaux de neurones sont beaucoup plus faibles que
celles des réseaux de neurones classiques.

L'activité de recherche du groupe \textbf{Spikes} consiste à
expérimenter de nouvelles techniques d'apprentissage pour ces
\textbf{réseaux de neurones impulsionnels} dans le but de contribuer à
les améliorer.

\hypertarget{les-simulateurs-de-neurones}{%
\section{Les simulateurs de
neurones}\label{les-simulateurs-de-neurones}}

J'ai participé au challenge "Dyni Odontocete Click Classification, 10 species [ DOCC10 ]
by Universite de Toulon" (https://challengedata.ens.fr/challenges/32) avec une partie de l'équipe Spike à savoir M.Clergue et M.Page. Le challenge consistait simplement à réaliser un classifieur qui classe les cachalots en une dizaine d'especes à partir de leurs "Clics" pour celà nous avions a disposition une base d'apprentissage labelisée ainsi qu'une base de test non labelisée sur laquelle nous pouvions évaluer les performances réelles de nos classifieurs. La principale difficulté que nous avons rencontré est le gap de performances de nos classifieurs entre la base labelisée que nous avions et la base non labelisée (en moyenne 20 pourcents de différences entre les deux). C'est pourquoi j'ai été chargé de mettre en place un ensemble d'outils permettant la visualisation de ces bases selons divers critéres et sous diverses formes.
