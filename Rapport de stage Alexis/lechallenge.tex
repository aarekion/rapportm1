\hypertarget{lechallenge}{%
\chapter{lechallenge}\label{lechallenge}}

Le challenge "Dyni Odontocete Click Classification, 10 species [ DOCC10 ]
by Universite de Toulon" (https://challengedata.ens.fr/challenges/32) consistait simplement à réaliser un classifieur qui classe les cachalots en une dizaine d'especes à partir de leurs "Clics" pour celà nous avions a disposition une base d'apprentissage labelisée ainsi qu'une base de test non labelisée sur laquelle nous pouvions évaluer les performances réelles de nos classifieurs. Pour cela nous labelision les exemples de la base de test puis envoyions nos prédictions sur le site du challenge qui nous renvoyais nos performances ainsi qu'un classement des performances de tous les participants.

Les deux bases sont consititués d'enregistrements audios des clics des différentes especes de cachalots que je présenterais plus en détail dans la partie analyse des données.

Pour résoudre ce probléme j'ai commencé par analyser les données afin
-D'avoir une vision claire du probléme a résoudre
-D'avoir des idées de méthodes de résolution du probléme
-Comprendre d'éventuelles incohérences dans nos futurs résultats (notament des différences de performances entre les deux bases)


Ensuite nous avons traité les données de différentes maniéres notament en faisant de la data augmentation et du traitement du signal.

Puis M. Clergue se chargeait de tester un multitude de techniques d'apprentissage sur les données ainsi traités.

M. Page nous a également fournis un grand nombre de petits programmes et de fonctions parfois simples parfois complexes dont nous avions besoin afin de réaliser nos différentes taches.
